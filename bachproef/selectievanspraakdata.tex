%%=============================================================================
%% Selectie van spraakdata
%%=============================================================================

\chapter{\IfLanguageName{dutch}{Selectie van spraakdata}{Selectie van spraakdata}}%
\label{ch:Selectie van spraakdata}

Nu de spraakmodellen geselecteerd zijn, is het tijd om de spraakdata te selecteren die gebruikt zullen worden om deze modellen te vergelijken. In dit hoofdstuk zullen de verschillende stappen worden toegelicht om de spraakdata te selecteren. Het is heel belangrijk dat er voldoende spraakdata worden geselecteerd die representatief zijn en een duidelijk beeld geven van de prestaties van de spraakmodellen. Er worden drie talen getest, namelijk Engels, Frans en Nederlands. Voor elk van deze talen zullen er verschillende soorten spraakdata worden verzameld, waaronder verschillende accenten en dialecten. Het is de bedoeling dat de spraakdata zowel luide als stille omgevingsgeluiden bevatten zodat de spraakmodellen in beide omstandigheden kunnen worden getest. Op deze manier kan men de accuraatheid van de spraakmodellen evalueren en vergelijken via de Word Error Rate (WER). Het is dus van groot belang dat de spraakdata zorgvuldig worden geselecteerd. Bij de selectie van de spraakdata is het ook heel belangrijk dat de spraakdata voldoet aan de gebruikersrechten en privacywetgeving. Wat betreft de privacywetgeving is het belangrijk dat de spraakdata anoniem is en dat er geen persoonlijke gegevens worden opgeslagen. Bij de spraakdata moet er ook een geschreven transcriptie aanwezig zijn zodat de resultaten van de spraakmodellen kunnen worden geëvalueerd.
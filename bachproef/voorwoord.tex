%%=============================================================================
%% Voorwoord
%%=============================================================================

\chapter*{\IfLanguageName{dutch}{Woord vooraf}{Preface}}%
\label{ch:voorwoord}

%% TODO:
%% Het voorwoord is het enige deel van de bachelorproef waar je vanuit je
%% eigen standpunt (``ik-vorm'') mag schrijven. Je kan hier bv. motiveren
%% waarom jij het onderwerp wil bespreken.
%% Vergeet ook niet te bedanken wie je geholpen/gesteund/... heeft

Met trots en voldoening presenteer ik u mijn bachelorproef, geschreven in het kader van mijn eindwerk voor de opleiding Toegepaste Informatica aan de Hogeschool Gent. Met deze proef heb ik getracht een antwoord te bieden op de vraag: ``Wat zijn de mogelijkheden en beperkingen bij het implementeren van geïntegreerde spraaktechnologie in uitdagende omgevingsfactoren?''.
\newline


De keuze voor dit onderwerp komt voort uit een verzoek vanuit het bedrijf waar ik mijn stage heb gelopen, delaware, om onderzoek te voeren naar de mogelijkheden van spraaktechnologie op smartphones. Deze technologie heeft de laatste jaren een enorme groei gekend door de opkomst van artificiële intelligentie en machine learning. Hierdoor is er veel interesse ontstaan bij ontwikkelaars om deze technologie te integreren in hun applicaties. Als software ontwikkelaar vind ik het belangrijk en interessant om me te verdiepen in deze technologieën en de mogelijkheden die ze bieden.
\newline


In dit voorwoord wil ik graag mijn dank uitspreken aan enkele personen die een cruciale rol hebben gespeeld in het succesvol afronden van mijn bachelorproef en die mij doorheen mijn studieperiode hebben ondersteund. Als eerste wil ik mijn co-promotor en promotor bedanken voor hun excelente begeleiding en ondersteuning. Hun feedback en advies hebben een waardevolle bijdrage geleverd aan de kwaliteit van mijn bachelorproef. Daarnaast wil ik ook mijn medestudenten bedanken voor de hulp en de leuke momenten gedurende de opleiding. Tot slot wil ik mijn ouders, vriendin en familie mijn oprechte dank betuigen voor hun onvoorwaardelijke steun en geloof in mij, bij alles wat ik onderneem.
\newline

Ik hoop dat deze bachelorproef een meerwaarde kan bieden voor de lezer en dat het een antwoord kan bieden op de vraagstelling. Ik wens u veel leesplezier toe.

%\lipsum[1-2]
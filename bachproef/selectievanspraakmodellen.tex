%%=============================================================================
%% Selectie van spraakmodellen
%% Long list & Short list
%%=============================================================================

\chapter{\IfLanguageName{dutch}{Selectie van spraakmodellen}{Selectie van spraakmodellen}}%
\label{ch:Selectie van spraakmodellen}

Nadat we een beter inzicht hebben verkregen in de verschillende fases van het onderzoek en de requirements hebben vastgelegd, is het tijd om de spraakmodellen te selecteren die zullen worden vergeleken. In dit hoofdstuk wordt er eerst een long list opgesteld van spraakmodellen die voldoen aan de vastgelegde requirements. Op basis van deze long list zullen we vervolgens een short list opstellen van de meest geschikte spraakmodellen.

\section{Long list}
In die long list zullen de spraakmodellen opgesomd worden die voldoen aan de vastgelegde requirements. Het MoSCoW-principe zal hierbij als leidraad dienen. De long list zal bestaan uit spraakmodellen die voldoen aan de meeste voorwaarden van de ``Must have`` en ``Should have`` requirements.

\begin{itemize}
  \item Google Cloud Speech-to-Text
  \item Microsoft Azure Speech-to-Text (Embedded)
  \item Mozilla DeepSpeech
  \item Kaldi
  \item PocketSphinx
  \item Vosk
  \item Whisper
  \item PicoVoice Cheetah
  \item IBM Watson Speech to Text
  \item Amazon Transcribe
  \item Baidu DeepSpeech
\end{itemize}

\section{Short list}
Na de long list is het tijd om een short list op te stellen van de meest geschikte spraakmodellen. Dit zal gebeuren in samenspraak met delaware.

\begin{itemize}
  \item Vosk
  \item Microsoft Azure Speech-to-Text (Embedded)
  \item Mozilla DeepSpeech
  \item Whisper
  \item PicoVoice Cheetah
\end{itemize}
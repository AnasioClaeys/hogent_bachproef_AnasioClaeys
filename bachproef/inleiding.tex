%%=============================================================================
%% Inleiding
%%=============================================================================

\chapter{\IfLanguageName{dutch}{Inleiding}{Introduction}}%
\label{ch:inleiding}

% De inleiding moet de lezer net genoeg informatie verschaffen om het onderwerp te begrijpen en in te zien waarom de onderzoeksvraag de moeite waard is om te onderzoeken. In de inleiding ga je literatuurverwijzingen beperken, zodat de tekst vlot leesbaar blijft. Je kan de inleiding verder onderverdelen in secties als dit de tekst verduidelijkt. Zaken die aan bod kunnen komen in de inleiding~\autocite{Pollefliet2011}:

% \begin{itemize}
%   \item context, achtergrond
%   \item afbakenen van het onderwerp
%   \item verantwoording van het onderwerp, methodologie
%   \item probleemstelling
%   \item onderzoeksdoelstelling
%   \item onderzoeksvraag
%   \item \ldots
% \end{itemize}
De aanleiding voor de onderzoeksvraag ontstond vanuit het Mobile, Web en IoT team van delaware. Ze hebben vastgesteld dat in de zakelijke wereld spraak-naar-teksttechnologieën steeds vaker worden ingezet voor diverse toepassingen in verschillende sectoren. Dit is te wijten aan de aanzienlijke vooruitgang van deze technologieën in de afgelopen jaren, gedreven door de ontwikkelingen in artificiële intelligentie en machine learning. Het team is specifiek geïnteresseerd in het onderzoeken van geïntegreerde spraaktechnologie, met name die functioneert zonder internetverbinding of met beperkte verwerkingscapaciteit op het apparaat.
%Dit bood mij de gelegenheid om mijn bachelorproef over dit onderwerp te schrijven.
\newline

Het doelapparaat voor het onderzoek is een smartphone uitgerust met een recente versie van het Android-besturingssysteem. Er zal eerst een literatuurstudie uitgevoerd worden waarbij er gekeken wordt naar de huidige stand van zaken en welke technologieën er beschikbaar zijn. Daarna zal een selectie van deze technologieën onderworpen worden aan een vergelijkend onderzoek. Dit houdt in dat de geselecteerde technologieën getest zullen worden op hun vermogen om spraak te herkennen in zowel stille als lawaaierige omstandigheden voor de talen Nederlands, Frans en Engels. De prestaties van de spraakherkenning zullen geëvalueerd worden op basis van de nauwkeurigheid, gemeten met de Word Error Rate (WER), en de snelheid van verwerking.
\newline

Het uiteindelijke resultaat van dit onderzoek is een proof-of-concept. Met behulp van de gekozen geïntegreerde spraakherkenningstechnologie uit voorgaande vergelijkende studie wordt een applicatie ontwikkeld waarmee een takenlijst kan worden beheerd door middel van spraakherkenning. Het doel is dat gebruikers via spraakopdrachten taken kunnen toevoegen, voltooien of verwijderen. Met deze applicatie kan er dan een conclusie worden getrokken over de effectiviteit van deze spraakmodellen en hoe goed deze presteren in de realiteit. 


\section{\IfLanguageName{dutch}{Probleemstelling}{Problem Statement}}%
\label{sec:probleemstelling}

% Uit je probleemstelling moet duidelijk zijn dat je onderzoek een meerwaarde heeft voor een concrete doelgroep. De doelgroep moet goed gedefinieerd en afgelijnd zijn. Doelgroepen als ``bedrijven,'' ``KMO's'', systeembeheerders, enz.~zijn nog te vaag. Als je een lijstje kan maken van de personen/organisaties die een meerwaarde zullen vinden in deze bachelorproef (dit is eigenlijk je steekproefkader), dan is dat een indicatie dat de doelgroep goed gedefinieerd is. Dit kan een enkel bedrijf zijn of zelfs één persoon (je co-promotor/opdrachtgever).

In het dagelijkse leven ontstaat vaak de behoefte om snel en efficiënt informatie vast te leggen. Personen met een motorische beperking, evenals degenen die moeite hebben met lezen of schrijven, kunnen baat hebben aan een oplossing met geïntegreerde spraak-naar-teksttechnologie. Deze kunnen ook nuttig zijn voor mensen die hun handen niet vrij hebben om een smartphone te bedienen. Een geïntegreerde oplossing biedt het voordeel van offline functionaliteit, waardoor de eindgebruiker te allen tijde en op elke locatie toegang heeft tot de technologie.


\section{\IfLanguageName{dutch}{Onderzoeksvraag}{Research question}}%
\label{sec:onderzoeksvraag}

% Wees zo concreet mogelijk bij het formuleren van je onderzoeksvraag. Een onderzoeksvraag is trouwens iets waar nog niemand op dit moment een antwoord heeft (voor zover je kan nagaan). Het opzoeken van bestaande informatie (bv. ``welke tools bestaan er voor deze toepassing?'') is dus geen onderzoeksvraag. Je kan de onderzoeksvraag verder specifiëren in deelvragen. Bv.~als je onderzoek gaat over performantiemetingen, dan 

Op dit moment bestaan er diverse spraak-naar-teksttechnologieën die een internetverbinding vereisen, terwijl slechts enkele daarvan ook offline functioneren. Momenteel is er nog geen duidelijk beeld van welke geïntegreerde technologieën het beste presteren in realistische omstandigheden. Hierdoor is de onderzoeksvraag van deze bachelorproef als volgt: ``Wat zijn de mogelijkheden en beperkingen bij het implementeren van geïntegreerde spraaktechnologie in uitdagende omgevingsfactoren?''.


\section{\IfLanguageName{dutch}{Onderzoeksdoelstelling}{Research objective}}%
\label{sec:onderzoeksdoelstelling}

% Wat is het beoogde resultaat van je bachelorproef? Wat zijn de criteria voor succes? Beschrijf die zo concreet mogelijk. Gaat het bv.\ om een proof-of-concept, een prototype, een verslag met aanbevelingen, een vergelijkende studie, enz.
Het doel van deze bachelorproef is om op basis van de literatuurstudie en vergelijkende studie een spraakherkenningstechnologie te selecteren die het meeste geschikt is voor het ontwikkelen van de proof-of-concept. Dit zal een applicatie zijn waarmee een takenlijst kan worden beheerd door middel van spraakherkenning. Aan de hand van de ontwikkelde applicatie zal er een conclusie getrokken worden en een antwoord geformuleerd worden op de onderzoeksvraag.


\section{\IfLanguageName{dutch}{Opzet van deze bachelorproef}{Structure of this bachelor thesis}}%
\label{sec:opzet-bachelorproef}

% Het is gebruikelijk aan het einde van de inleiding een overzicht te
% geven van de opbouw van de rest van de tekst. Deze sectie bevat al een aanzet
% die je kan aanvullen/aanpassen in functie van je eigen tekst.

De rest van deze bachelorproef is als volgt opgebouwd:

In Hoofdstuk~\ref{ch:stand-van-zaken} wordt een overzicht gegeven van de stand van zaken binnen het onderzoeksdomein, op basis van een literatuurstudie.

In Hoofdstuk~\ref{ch:methodologie} wordt de methodologie toegelicht en worden de gebruikte onderzoekstechnieken besproken om een antwoord te kunnen formuleren op de onderzoeksvragen.

% TODO: Vul hier aan voor je eigen hoofstukken, één of twee zinnen per hoofdstuk

In Hoofdstuk~\ref{ch:conclusie}, tenslotte, wordt de conclusie gegeven en een antwoord geformuleerd op de onderzoeksvragen. Daarbij wordt ook een aanzet gegeven voor toekomstig onderzoek binnen dit domein.
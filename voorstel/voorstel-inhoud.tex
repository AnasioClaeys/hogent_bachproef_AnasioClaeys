%---------- Inleiding ---------------------------------------------------------

\section{Introductie}%
\label{sec:introductie}
% Waarover zal je bachelorproef gaan? Introduceer het thema en zorg dat volgende zaken zeker duidelijk aanwezig zijn:

% \begin{itemize}
%   \item kaderen thema
%   \item de doelgroep
%   \item de probleemstelling en (centrale) onderzoeksvraag
%   \item de onderzoeksdoelstelling
% \end{itemize}

% Denk er aan: een typische bachelorproef is \textit{toegepast onderzoek}, wat betekent dat je start vanuit een concrete probleemsituatie in bedrijfscontext, een \textbf{casus}. Het is belangrijk om je onderwerp goed af te bakenen: je gaat voor die \textit{ene specifieke probleemsituatie} op zoek naar een goede oplossing, op basis van de huidige kennis in het vakgebied.

% De doelgroep moet ook concreet en duidelijk zijn, dus geen algemene of vaag gedefinieerde groepen zoals \emph{bedrijven}, \emph{developers}, \emph{Vlamingen}, enz. Je richt je in elk geval op it-professionals, een bachelorproef is geen populariserende tekst. Eén specifiek bedrijf (die te maken hebben met een concrete probleemsituatie) is dus beter dan \emph{bedrijven} in het algemeen.

% Formuleer duidelijk de onderzoeksvraag! De begeleiders lezen nog steeds te veel voorstellen waarin we geen onderzoeksvraag terugvinden.

% Schrijf ook iets over de doelstelling. Wat zie je als het concrete eindresultaat van je onderzoek, naast de uitgeschreven scriptie? Is het een proof-of-concept, een rapport met aanbevelingen, \ldots Met welk eindresultaat kan je je bachelorproef als een succes beschouwen?

%---------- Stand van zaken ---------------------------------------------------
De aanleiding voor de onderzoeksvraag ontstond vanuit het Mobile, Web en IoT-team van Delaware Consulting, dat graag onderzoek wilde doen naar de mogelijkheden van geïntegreerde spraaktechnologie. Dit is een technologie die spraakherkenning mogelijk maakt op apparaten met beperkte middelen, zoals bij afwezigheid van een internetverbinding. Met dit onderzoek wil Delaware Consulting een beter beeld krijgen van de mogelijkheden en beperkingen van geïntegreerde spraaktechnologie met als uiteindelijk \\doel dat het bedrijf in de toekomst een betere en efficiëntere service kan bieden aan zijn klanten.

Spraaktechnologie is de laatste jaren sterk geëvolueerd door de opkomst van AI en machine learning. Het is een actueel onderwerp dat veel aandacht krijgt in de media. Deze technologie kan een meerwaarde bieden voor specifieke applicaties in verschillende sectoren. Voordat deze technologie toegepast wordt, moet er eerst onderzocht worden welke spraakmodellen er bestaan en hoe die presteren onder verschillende omstandigheden. Aan de hand van deze probleemstelling is de onderzoeksvraag ``Wat zijn de mogelijkheden en beperkingen bij het implementeren van geïntegreerde spraaktechnologie in uitdagende omgevingsfactoren?'' ontstaan.

Het uiteindelijke resultaat van dit onderzoek is een proof-of-concept. Met behulp van de gekozen geïntegreerde spraakherkenningstechnologie uit de literatuurstudie wordt een applicatie ontwikkeld waarmee een takenlijst kan worden beheerd door middel van spraakherkenning. Hier komt zowel tekst-naar-spraak als spraak-naar-\\tekst aan bod. Met deze applicatie kan er dan een conclusie worden getrokken over de effectiviteit van deze spraakmodellen en hoe goed deze presteren in de realiteit. 



\section{Literatuurstudie}%
\label{sec:literatuurstudie}

Door de opkomst van artificiële intelligentie en machine learning is spraakherkenningstechnologie de laatste jaren sterk geëvolueerd. Spraakherkenning is een technologische mogelijkheid waarmee een softwareprogramma in staat is om gesproken menselijke taal om te zetten in geschreven tekst en omgekeerd. Het maakt het mogelijk om apparaten handsfree te bedienen en input automatisch te gaan vertalen. Om deze technologie te kunnen toepassen, moet er een microfoon aanwezig zijn op het apparaat om de trillingen van een stem om te zetten in een elektrisch signaal. Vervolgens wordt dit signaal omgezet in een digitaal signaal dat door een spraakherkenningsprogramma kan geanalyseerd worden.\autocite{Zwass2022}

Er zijn heel wat voordelen bij het gebruik van spraakherkenningstechnologie. Het is een snellere manier om tekst in te voeren dan bij het gebruik van een toetsenbord. Het is ook een handige tool voor mensen die problemen hebben met spreken of schrijven doordat deze spraakmodellen in staat zijn om van tekst-naar-spraak als van spraak-naar-tekst te gaan. Er zijn ook enkele nadelen aan verbonden. Zo is het niet altijd even nauwkeurig en kan het soms moeilijk zijn om de juiste woorden te herkennen. In het huidig stadium van de technologie is er controle nodig voor het corrigeren van fouten. \autocite{RingCentral2021}

Spraakherkenningsystemen hebben echter \\enkele uitdagingen. Het moeilijkste aspect is de taaldekking voor de spraakmodellen doordat er heel veel verschillende talen bestaan en elke taal zijn eigen accenten en dialecten heeft. Een ander belangrijk spraakherkenningsprobleem is achtergrondlawaai doordat dit mee wordt opgenomen door de microfoon. Dit kan de spraakherkenning verstoren en de nauwkeurigheid van het spraakmodel verminderen. Bij het gebruik van spraakherkenningstechnologie is het belangrijk om rekening te houden met de privacy van de gebruiker, wat een essentieel aandachtspunt is tijdens de implementatie. \autocite{Singh2022}

De bekendste spraakherkenningstechno-\\logieën zijn: Microsoft Azure Cognitive Speech, Google Cloud Speech-to-Text, Amazon Transcribe, IBM Watson, Mozilla DeepSpeech en Whisper.ai. Deze spraakmodellen zijn bekend omdat ze afkomstig zijn van grote bedrijven met voldoende middelen om deze technologieën te ontwikkelen. Een aantal van deze spraakmodellen zijn cloud-gebaseerd en hebben dus een internetverbinding nodig om te kunnen werken. Bepaalde spraakmodellen bieden ook functionaliteiten voor offline gebruik door middel van een SDK die op het apparaat geïnstalleerd kan worden. Elk model heeft zijn eigen voor- en nadelen, waardoor je ze kunt gaan vergelijken. De keuze van het model is afhankelijk van de behoeften van het project. \autocite{Fox2023}

Microsoft Azure Cognitive Speech is een platform dat zich richt op de ontwikkeling van spraakherkenningstechnologie. De Text-to-Speech en Speech-to-Text modellen zijn onderdelen van dit platform waarbij beide modellen een cloud-\\gebaseerde API hebben die een internetverbinding nodig heeft om te kunnen werken. Via het installeren van een SDK is het mogelijk om apparaten offline te laten werken met deze modellen. Op dit platform zijn ook andere spraakmodellen aanwezig zoals Translation en Speaker Recognition. \autocite{Depypere2023}

Google Cloud Speech-to-Text is een spraakherkenningstechnologie die ontwikkeld is door Google. Het ondersteunt meer dan 120 talen en dialecten. Voor de implementatie van deze technologie is een cloud-gebaseerde API vereist die een internetverbinding nodig heeft. Een nadeel is dat dit model niet zonder internetverbinding kan werken. Deze technologie is één van de meest nauwkeurige spraakmodellen die er bestaan op dit moment. Het biedt heel wat features aan zoals het mogelijk maken om een eigen custom spraakmodel te trainen of het herkennen van verschillende sprekers. \autocite{Wang2021}

Amazon Transcribe maakt het mogelijk om spraak om te zetten naar tekst en omgekeerd. Het model biedt voor meerdere talen ondersteuning aan waarbij het mogelijk is om realtime transcriptie toe te passen. Het model is cloud-\\gebaseerd en heeft dus een internetverbinding nodig om te kunnen werken. Dit model heeft een hoge nauwkeurigheid en is in staat om achtergrondlawaai te filteren. \autocite{Kumbhar2023}

Mozilla DeepSpeech is een open-source spraakherkenningstechnologie die is ontwikkeld door Mozilla, de organisatie achter de welbekende \\browser Firefox. Dit model staat bekend om zijn offline functionaliteit en is volledig gratis. Het model moet wel getraind worden om een goede nauwkeurigheid te hebben. Het model is nog in ontwikkeling, dus het is niet zo nauwkeurig en snel als de voorgaande modellen. Het ondersteunt enkel spraak naar tekst en niet omgekeerd. \autocite{Tang2022}

Whisper.ai is een spraakherkenningstechnologie dat ontwikkeld is door het welgekende OpenAI. Dit bedrijf is bekend door zijn onderzoek naar artificiële intelligentie. De technologie is open-source en kan volledig offline werken. Het heeft verschillende modellen waarbij je kunt kiezen tussen modellen met een hogere nauwkeurigheid of een snellere verwerkingstijd door de grootte van het model aan te passen. Het model biedt ondersteuning aan voor meerdere talen die een eigen score hebben gekregen op basis van de nauwkeurigheid. Deze score is gebaseerd op de WER (Word Error Rate) die aangeeft hoeveel woorden er fout zijn herkend. \autocite{OpenAI2023}

Om de spraakherkenningsmodellen te vergelijken, zijn er verschillende methoden beschikbaar om dit te doen. Woordfoutpercentage (WER) is bekende methode die de fouten die gemaakt worden bij woorden, tussen de input en de output meet en gaat weergeven aan de hand van een percentage. De snelheid van een model kan ook gemeten worden door de verwerkingstijd te meten bij het uitvoeren van dezelfde dataset.\\ \autocite{OConnor2023}


% Hier beschrijf je de \emph{state-of-the-art} rondom je gekozen onderzoeksdomein, d.w.z.\ een inleidende, doorlopende tekst over het onderzoeksdomein van je bachelorproef. Je steunt daarbij heel sterk op de professionele \emph{vakliteratuur}, en niet zozeer op populariserende teksten voor een breed publiek. Wat is de huidige stand van zaken in dit domein, en wat zijn nog eventuele open vragen (die misschien de aanleiding waren tot je onderzoeksvraag!)?

% Je mag de titel van deze sectie ook aanpassen (literatuurstudie, stand van zaken, enz.). Zijn er al gelijkaardige onderzoeken gevoerd? Wat concluderen ze? Wat is het verschil met jouw onderzoek?

% Verwijs bij elke introductie van een term of bewering over het domein naar de vakliteratuur, bijvoorbeeld~\autocite{Hykes2013}! Denk zeker goed na welke werken je refereert en waarom.

% Draag zorg voor correcte literatuurverwijzingen! Een bronvermelding hoort thuis \emph{binnen} de zin waar je je op die bron baseert, dus niet er buiten! Maak meteen een verwijzing als je gebruik maakt van een bron. Doe dit dus \emph{niet} aan het einde van een lange paragraaf. Baseer nooit teveel aansluitende tekst op eenzelfde bron.

% Als je informatie over bronnen verzamelt in JabRef, zorg er dan voor dat alle nodige info aanwezig is om de bron terug te vinden (zoals uitvoerig besproken in de lessen Research Methods).

% % Voor literatuurverwijzingen zijn er twee belangrijke commando's:
% % \autocite{KEY} => (Auteur, jaartal) Gebruik dit als de naam van de auteur
% %   geen onderdeel is van de zin.
% % \textcite{KEY} => Auteur (jaartal)  Gebruik dit als de auteursnaam wel een
% %   functie heeft in de zin (bv. ``Uit onderzoek door Doll & Hill (1954) bleek
% %   ...'')

% Je mag deze sectie nog verder onderverdelen in subsecties als dit de structuur van de tekst kan verduidelijken.



%---------- Methodologie ------------------------------------------------------
\section{Methodologie}%
\label{sec:methodologie}

Het onderzoek is opgedeeld in vijf opeenvolgende fases. In de eerste fase wordt informatie verzameld over de verschillende spraakmodellen en hun werking, met focus op modellen die met beperkte middelen kunnen werken. Deze informatie zal worden verzameld door een grondige studie van vakliteratuur. De geschatte duurtijd van deze fase bedraagt een tweetal weken. 

In de daaropvolgende fase zal een vergelijkende studie worden uitgevoerd waarbij de verschillende spraakmodellen worden vergeleken op basis van hun prestaties door met deze modellen te experimenten. Hierbij wordt gekeken naar hun nauwkeurigheid en prestaties in verschillende omgevingsfactoren voor de talen Nederlands, Frans en Engels. Op basis van deze vergelijking wordt er een spraakmodel gekozen dat zal worden gebruikt voor de proof-of-concept. De geschatte duurtijd van deze fase bedraagt een drietal weken.

In de derde fase wordt een applicatieontwerp gecreëerd, waarbij wireframes worden gebruikt om de gebruikersinterface visueel weer te geven. Deze fase zal naar verwachting ongeveer 1 week duren.

In de vierde fase wordt de implementatie van de applicatie uitgevoerd. Hierbij wordt de gekozen spraaktechnologie geïntegreerd in de applicatie. De geschatte duurtijd van deze fase bedraagt een tweetal weken.

In de afsluitende fase van het onderzoek wordt de geïmplementeerde proof-of-concept \\beoordeeld en geanalyseerd in de realiteit. Het doel is om vast te stellen in hoeverre de gestelde doelstellingen zijn behaald en om inzichten te verkrijgen voor eventuele verbeteringen of aanpassingen. Hierbij wordt een evaluatierapport ontwikkeld waarin de verzamelde analyse van de resultaten wordt samengevat en conclusies worden getrokken met betrekking tot de bruikbaarheid en effectiviteit van het toegepaste spraakmodel. De geschatte duurtijd van deze fase bedraagt een tweetal weken.


% Hier beschrijf je hoe je van plan bent het onderzoek te voeren. Welke onderzoekstechniek ga je toepassen om elk van je onderzoeksvragen te beantwoorden? Gebruik je hiervoor literatuurstudie, interviews met belanghebbenden (bv.~voor requirements-analyse), experimenten, simulaties, vergelijkende studie, risico-analyse, PoC, \ldots?

% Valt je onderwerp onder één van de typische soorten bachelorproeven die besproken zijn in de lessen Research Methods (bv.\ vergelijkende studie of risico-analyse)? Zorg er dan ook voor dat we duidelijk de verschillende stappen terug vinden die we verwachten in dit soort onderzoek!

% Vermijd onderzoekstechnieken die geen objectieve, meetbare resultaten kunnen opleveren. Enquêtes, bijvoorbeeld, zijn voor een bachelorproef informatica meestal \textbf{niet geschikt}. De antwoorden zijn eerder meningen dan feiten en in de praktijk blijkt het ook bijzonder moeilijk om voldoende respondenten te vinden. Studenten die een enquête willen voeren, hebben meestal ook geen goede definitie van de populatie, waardoor ook niet kan aangetoond worden dat eventuele resultaten representatief zijn.

% Uit dit onderdeel moet duidelijk naar voor komen dat je bachelorproef ook technisch voldoen\-de diepgang zal bevatten. Het zou niet kloppen als een bachelorproef informatica ook door bv.\ een student marketing zou kunnen uitgevoerd worden.

% Je beschrijft ook al welke tools (hardware, software, diensten, \ldots) je denkt hiervoor te gebruiken of te ontwikkelen.

% Probeer ook een tijdschatting te maken. Hoe lang zal je met elke fase van je onderzoek bezig zijn en wat zijn de concrete \emph{deliverables} in elke fase?

%---------- Verwachte resultaten ----------------------------------------------
\section{Verwacht resultaat, conclusie}%
\label{sec:verwachte_resultaten}

% Hier beschrijf je welke resultaten je verwacht. Als je metingen en simulaties uitvoert, kan je hier al mock-ups maken van de grafieken samen met de verwachte conclusies. Benoem zeker al je assen en de onderdelen van de grafiek die je gaat gebruiken. Dit zorgt ervoor dat je concreet weet welk soort data je moet verzamelen en hoe je die moet meten.

% Wat heeft de doelgroep van je onderzoek aan het resultaat? Op welke manier zorgt jouw bachelorproef voor een meerwaarde?

% Hier beschrijf je wat je verwacht uit je onderzoek, met de motivatie waarom. Het is \textbf{niet} erg indien uit je onderzoek andere resultaten en conclusies vloeien dan dat je hier beschrijft: het is dan juist interessant om te onderzoeken waarom jouw hypothesen niet overeenkomen met de resultaten.

Er wordt verwacht dat de geïmplementeerde proof-of-concept een applicatie is waarmee tijd kan worden bespaard bij het beheren van een takenlijst. Personen met visuele of motorische beperkingen, evenals degenen die moeite hebben met typen of schrijven zullen baat hebben bij deze oplossing. 

Op basis van het evaluatierapport zouden de uitdagingen in kaart moeten worden gebracht die zich voordoen bij het gebruik van deze geïntegreerde spraaktechnologie. Er wordt verwacht dat de uitdagingen vooral zullen liggen bij de taalkeuze en de omgevingsfactoren.

De ontwikkeling van het proof-of-concept kan bijdragen aan het begrijpen en uitbreiden van spraaktechnologie. In de toekomst zal spraaktechnologie met geavanceerde AI en machine learning verder worden verbeterd, wat kan leiden tot betere prestaties en meer toegevoegde waarde in andere sectoren.
